\chapter{简介}

战争冲突中,受伤的总是民众。本项目旨在从人道主义出发,搭建帮助难民的互助救助系统。本次主要关注的是俄乌冲突中的受伤民众,他们的安危是俄乌两国在谈判中的议题之一,也收到了国际社会的关注。
\\
\section{背景}

战争的出现是大国之间的博弈和多方利益的考量,在战争中民众的生命安全一直是参战双方的谈判议题,以近期出现的俄乌冲突为例,早在俄乌之间的战争还未爆发的前一周,乌克兰的前一百名富豪们就有96名乘包机逃离了乌克兰,在战时状态,剩下的乌克兰老百姓不仅不被允许离境,还发放枪支给他们,希望他们抵抗。甚至连部分女性,也被迫成为士兵,派往战场。这些人中,很多人都是在两天的时间内才学会射击。

一个逃难去到波兰的乌克兰难民小男孩在被问道,为什么不去选一件为难民孩子们捐赠的玩具,反而一个人在这里孤独悲伤地坐着?他回答说:“小男孩?我现在是家里的大人了”。

在冲突的民众,主要面临以下问题:

\begin{itemize}
\item 缺少食物,淡水,住所等基本生存需求。
\item 缺少基本应急药品。
\item 与临近国家语言不通。
\item 无法与家人团聚 。
\end{itemize}

解决这些问题也是我们开发设计这套系统首要关注的内容。
\\
\section{文档概要}

该文档是针对该难民救助系统的设计文档,包括有软件架构设计、高层设计、接口说明、业务对象及相互关系、详细设计。

\begin{itemize}
  \item 软件架构设计
  
  本系统采用了微服务架构,将业务拆分为了多个子系统独立部署,同时结合当前流行的微服务技术如网关、注册中心等。
  \item 高层设计
  
  本系统采用传统MVC三层模式,将各个子模块分为视图层、业务层和持久层,其中视图层负责和前端进行数据交互,业务层负责核心逻辑处理,持久层负责和数据库进行数据交互。同时该系统还拆分为多个子系统,以达到松耦合、易扩展的目的。

  \item 接口说明
  
  将对系统中用到的interface和abstract class进行详细说明,包括他们的作用、public方法的用途以及参数的说明。

  \item 业务对象及相互关系
  
  将对本系统涉及到的业务对象,如用户类、权限类、新闻类、帖子类等,进行详细的说明,并对其相互关系进行设计。

  \item 详细设计
  
  将对本系统的数据库、数据表关系以及各个模块的业务流程进行详细设计。
  
\end{itemize}


\section{约束}

\begin{itemize}
  \item 相关需要有身份认证以保证安全性。
  \item 系统无前置的软硬件基础。
\end{itemize}
\newpage
\section{参考资料}
\begin{table}[htbp]
  \centering
  \caption{参考资料 }
  \vspace{0.5em}\wuhao
  \begin{tabular}{|c|c|c|c|c|}
    \hline
    \makebox[0.2\textwidth][c]{资料名称} & \makebox[0.2\textwidth][c]{版本日期} & \makebox[0.2\textwidth][c]{说明} \\
    \hline
    UML大战需求分析                      & 2012年2月第一版                      & 无                               \\
    \hline
    待补充                               &                                      &                                  \\
    \hline
  \end{tabular}
\end{table}

