\chapter{接口说明}
本章将对系统中的应用的接口,包括interface,abstract class进行详细说明,主要在于说明他们的public方法,以及方法中参数的意义和返回值的意义。

\section{Security模块}
为了便于扩展不同的认证授权方案,Security模块定义了许多认证授权操作的接口,以及相关过滤器的抽象类,包括登录过滤器、授权过滤器、用户加载接口等。
\subsection{UsernamePasswordAuthenticationFilter}
此抽象类为登录时调用的过滤器类。类中共定义了3个主要方法,作用分别为用户认证、认证成功和认证失败。
\begin{itemize}
    \item attemptAuthentication方法,返回类型为Authentication(用户认证类),参数有2个,类型分别为HttpServletRequest、HttpServletResponse。方法用途为将request中的用户信息进行校验。
    \item successfulAuthentication方法,返回类型为void,参数有4个,类型分别为HttpServletRequest、HttpServletResponse、Authentication、FilterChain(过滤链)。方法用途为用户认证成功后进行相关操作,包括加载权限等,然后将用户认证信息通过过滤链传递给下个过滤器。
    \item unsuccessfulAuthentication方法,返回类型为void,参数有3个,类型分别为HttpServletRequest、HttpServletResponse、Authentication。方法用途为用户认证失败后进行相关操作,包括返回失败信息等。
\end{itemize}

\subsection{BasicAuthenticationFilter}
此抽象类为授权时调用的过滤器类,类中需要说明的方法主要有1个,作用是获取用户权限,然后交给鉴权过滤器去鉴权。
\begin{itemize}
    \item doFilterInternal方法,返回类型为void,参数有3个,类型分别为HttpServletRequest、HttpServletResponse、FilterChain。方法用途为将request中的用户进行权限加载,然后把权限信息交给鉴权过滤器去鉴权。
\end{itemize}

\subsection{AccessDeniedHandler}
此接口定义了用户权限不够时所调用方法。主要的方法有1个,作用是设置用户权限不足的信息,然后将其返回给用户。
\begin{itemize}
    \item handle方法,返回类型为void,参数有3个,类型分别为HttpServletRequest、HttpServletResponse、AccessDeniedException(权限异常类)。方法用途为在HttpServletResponse中设置用户权限不足的信息并返回给用户。
\end{itemize}

\subsection{AuthenticationEntryPoint}
此接口定义了用户认证失败时所调用方法。主要的方法有1个,作用是设置用户认证失败的信息,然后将其返回给用户。
\begin{itemize}
    \item commence方法,返回类型为void,参数有3个,类型分别为HttpServletRequest、HttpServletResponse、AuthenticationException(认证异常类)。方法用途为在HttpServletResponse中设置用户认证失败的信息并返回给用户。
\end{itemize}

\subsection{PasswordEncoder}
此接口定义了用户密码加密和匹配的策略,主要有2个方法,作用分别是用户密码的加密算法和用户密码的匹配算法。
\begin{itemize}
    \item encode方法,返回类型为String,参数有1个,类型为CharSequence(用户传来的原始密码),作用在于对用户输入的密码进行加密计算,至于加密密码可以自定义,比如md5加密等。
    \item matches方法,返回类型为Boolean,参数有2个,类型都是CharSequence,分别表示用户输入的密码和数据库查询的密码,然后通过自定义的匹配算法进行匹配。
\end{itemize}

\subsection{UserDetailsService}
此接口定义了用户详细信息的加载方法,主要是1个方法,作用为从数据库加载用户的基本信息、角色信息和权限信息,并进行用户登录初始化工作。
\begin{itemize}
    \item loadUserByUsername方法,返回类型为UserDetail,参数有1个,类型分别为String,表示登录用户的用户名。方法用途为根据传入的用户名从数据库加载用户基本信息、角色信息和权限信息,并进行初始化工作,比如权限信息存入Redis缓存等。
\end{itemize}

\section{GateWay模块}
因为GateWay网关是需要处理客户端发来的一切请求,我们需要自定义一些过滤器。为了可扩展性,需要一个全局的过滤器类,实现过滤器的扩展。
\subsection{GlobalFilter}
此接口定义了过滤器实际过滤采用的方法,主要方法有1个,作用是接收并预处理客户端发来的一切请求。
\begin{itemize}
    \item filter方法,返回类型为Mono<void>,参数有2个,类型分别为ServerWebExchange(客户端请求)、GatewayFilterChain(网关过滤器链)。方法用途为将客户端请求进行预处理,如果符合要求则放行,否则返回错误信息。
\end{itemize}

\section{Service模块}
Service是业务模块,用的是MVC的框架,其中将每个具体的业务方法都进行了interface+impl的组合,即每个业务方法都被抽象成了一个接口,这样的话
我们需要进行扩展就只需要新建impl即可,增加了系统的可扩展性。

