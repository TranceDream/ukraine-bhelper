\chapter{简介}

战争冲突中,受伤的总是民众。本项目旨在从人道主义出发,搭建帮助难民的互助救助系统。本次主要关注的是俄乌冲突中的受伤民众,他们的安危是俄乌两国在谈判中的议题之一,也收到了国际社会的关注。

\subsection{背景}

战争的出现是大国之间的博弈和多方利益的考量,在战争中民众的生命安全一直是参战双方的谈判议题,以近期出现的俄乌冲突为例,早在俄乌之间的战争还未爆发的前一周,乌克兰的前一百名富豪们就有96名乘包机逃离了乌克兰,在战时状态,剩下的乌克兰老百姓不仅不被允许离境,还发放枪支给他们,希望他们抵抗。甚至连部分女性,也被迫成为士兵,派往战场。这些人中,很多人都是在两天的时间内才学会射击。

一个逃难去到波兰的乌克兰难民小男孩在被问道,为什么不去选一件为难民孩子们捐赠的玩具,反而一个人在这里孤独悲伤地坐着?他回答说:“小男孩?我现在是家里的大人了”。
	
在冲突的民众,主要面临以下问题:

\begin{itemize}
   \item 缺少食物,淡水,住所等基本生存需求。
   \item 缺少基本应急药品。
   \item 与临近国家语言不通。
   \item 无法与家人团聚。
\end{itemize}

\subsection{定义、缩略语}

\subsection{约束}

\subsection{参考资料}

