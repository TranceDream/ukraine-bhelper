\chapter{简介}

战争冲突中,受伤的总是民众。本项目旨在从人道主义出发,搭建帮助难民的互助救助系统。本次主要关注的是俄乌冲突中的受伤民众,他们的安危是俄乌两国在谈判中的议题之一,也收到了国际社会的关注。
\\
\section{背景}

战争的出现是大国之间的博弈和多方利益的考量,在战争中民众的生命安全一直是参战双方的谈判议题,以近期出现的俄乌冲突为例,早在俄乌之间的战争还未爆发的前一周,乌克兰的前一百名富豪们就有96名乘包机逃离了乌克兰,在战时状态,剩下的乌克兰老百姓不仅不被允许离境,还发放枪支给他们,希望他们抵抗。甚至连部分女性,也被迫成为士兵,派往战场。这些人中,很多人都是在两天的时间内才学会射击。

一个逃难去到波兰的乌克兰难民小男孩在被问道,为什么不去选一件为难民孩子们捐赠的玩具,反而一个人在这里孤独悲伤地坐着?他回答说:“小男孩?我现在是家里的大人了”。

在冲突的民众,主要面临以下问题:

\begin{itemize}
\item 缺少食物,淡水,住所等基本生存需求。
\item 缺少基本应急药品。
\item 与临近国家语言不通。
\item 无法与家人团聚 。
\end{itemize}

解决这些问题也是我们开发设计这套系统首要关注的内容。
\\
\section{定义、缩略语}

\begin{table}[htbp]
  \centering
  \caption{定义、缩略语}
  \vspace{0.5em}\wuhao
  \begin{tabular}{|c|c|}
    \hline
    \makebox[0.2\textwidth][c]{术语} & \makebox[0.2\textwidth][c]{解释}   \\
    \hline
    无                               & 无                                \\
    \hline
  \end{tabular}
\end{table}

\section{约束}

\begin{itemize}
  \item 相关需要有身份认证以保证安全性。
  \item 系统无前置的软硬件基础。
\end{itemize}
\newpage
\section{参考资料}
\begin{table}[htbp]
  \centering
  \caption{参考资料 }
  \vspace{0.5em}\wuhao
  \begin{tabular}{|c|c|c|c|c|}
    \hline
    \makebox[0.2\textwidth][c]{资料名称} & \makebox[0.2\textwidth][c]{版本日期} & \makebox[0.2\textwidth][c]{说明} \\
    \hline
    UML大战需求分析                      & 2012年2月第一版                      & 无                               \\
    \hline
    待补充                               &                                      &                                  \\
    \hline
  \end{tabular}
\end{table}

